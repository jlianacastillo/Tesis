 \documentclass[hidelinks]{Documento}

\setcounter{secnumdepth}{5}
\setcounter{tocdepth}{5}

\usepackage{tabularx}
\usepackage{graphicx,scalerel}
\usepackage[T1]{fontenc}
\usepackage{mathptmx}
\usepackage[labelsep=period]{caption}
\usepackage{titlesec}
\usepackage{fancyhdr}

\addto{\captionsspanish}{\renewcommand{\bibname}{Referencias}}
\addto{\captionsspanish}{\renewcommand{\refname}{Referencias}}

\newcommand\fsbullet[1][.8]{\mathbin{\ThisStyle{\vcenter{\hbox{%
  \scalebox{#1}{$\SavedStyle\bullet$}}}}}}
\newcommand\ssbullet[1][.6]{\mathbin{\ThisStyle{\vcenter{\hbox{%
  \scalebox{#1}{$\SavedStyle\bullet$}}}}}}
\newcommand\tsbullet[1][.4]{\mathbin{\ThisStyle{\vcenter{\hbox{%
  \scalebox{#1}{$\SavedStyle\bullet$}}}}}}
\newcommand\fosbullet[1][.3]{\mathbin{\ThisStyle{\vcenter{\hbox{%
  \scalebox{#1}{$\SavedStyle\bullet$}}}}}}
\renewcommand{\labelitemi}{\(\fsbullet\)}
\renewcommand{\labelitemii}{\(\ssbullet\)}
\renewcommand{\labelitemiii}{\(\tsbullet\)}
\renewcommand{\labelitemiv}{\(\fosbullet\)}%
\renewcommand{\rmdefault}{ptm}
\renewcommand{\headrulewidth}{0pt}

\DeclareGraphicsExtensions{.pdf,.png,.jpg}
\graphicspath{{./figuras/}}

\fancyhf{}
\fancyfoot[R]{\thepage}

\fancypagestyle{plain}{%
  \fancyhf{}
  \fancyfoot[C]{\thepage}
}

\pagestyle{fancy}
\usepackage{blindtext}

\counterwithout{figure}{chapter}
\counterwithout{table}{chapter}
\counterwithout{equation}{chapter}

\titleformat{\subparagraph}
    {\normalfont\normalsize\bfseries}{\thesubparagraph}{1em}{}
\titlespacing*{\subparagraph}{\parindent}{3.25ex plus 1ex minus .2ex}{.75ex plus .1ex}

\titulo{``Desarrollo de aplicación web "Network WE" como método de comunicación internacional para promover la igualdad de género en UABJB y UDEC.''}
\fecha{FEBRERO, 2023}
\autor{JULIANA CASTILLO ARAUJO}
\grado{INGENIERO EN SISTEMAS COMPUTACIONALES}
\asesorA{Nombres Apellidos}
\asesorB{Nombres Apellidos}
\modalidad{TESIS}
\carrera{INGENIERÍA EN SISTEMAS COMPUTACIONALES}
\autorizacion{autorizacion.jpg}

\begin{document}

\maketitle
\frontmatter
\myemptypage
\insertarautorizacion 

\begin{agradecimientos}
\begin{center}

\textbf{Agradecimientos}
\end{center}
Agradezco a las instituciones educativas la Universidad de Cundinamarca y la Universidad Autonoma del Beni José Ballivian por contribuir en el conocimiento ingenieril e igualitorio en el desarrollo de este proyecto internacional
\end{agradecimientos}
\addcontentsline{toc}{section}{Agradecimientos}

\begin{dedicatoria}
\begin{center}
Dedicado a Ana Hurtado Mesa, Mary Rubiano Acosta, Yesica Gómez Ortiz, Adriana Adriana Alvarez Flores
\end{center}
\end{dedicatoria}
\addcontentsline{toc}{section}{Dedicatoria}

\begin{abstract}
Se realizó una investigación en conjunto con dos universidades, la Universidad Autónoma del Beni “José Ballivian” (UABJB) y la Universidad de Cundinamarca (UDEC) para el correcto desarrollo de una aplicación web para la red de conocimiento Network WE (Women Engineering), con el objetivo de promover la igualdad de género en el sector tecnológico e ingenieril en ambas universidades durante el mes de Agosto del 2022 haciendo un uso del levantamiento de requerimientos en software en las fases de la metodología  XP (Extreme Programming) que permitió desarrollar una investigación aplicada en el nivel cuantitativo y cualitativo beneficiando a las estudiantes nacionales e internacionales de ingeniería de sistemas.
\end{abstract}
\addcontentsline{toc}{section}{Resumen}
\selectlanguage{english}
\begin{abstract}
An investigation was carried out in conjunction with two universities, the Autonomous University of Beni "José Ballivian" (UABJB) and the University of Cundinamarca (UDEC) for the correct development of a web application for the Network WE (Women Engineering) knowledge network. with the objective of promoting gender equality in the technological and engineering sector in both universities during the month of August 2022, making use of the survey of software requirements in the phases of the XP (Extreme Programming) methodology that allowed the development of an investigation applied at the quantitative and qualitative level, benefiting national and international systems engineering students.


\end{abstract}
\selectlanguage{spanish}
\addcontentsline{toc}{section}{Abstract}

\tableofcontents
\addcontentsline{toc}{section}{Índice general}
{\noskip\listoffigures} 
\addcontentsline{toc}{section}{Índice de figuras}
{\noskip\listoftables} 


\mainmatter

\fontfamily{ptm}\selectfont
\chapter{Introducción}
La comunicación es la parte esencial del conocimiento, con ella podemos transmitir información que a menudo es proyectada hacia ambientes de investigación. Un grupo de personas que interactúan entre sí produciendo y socializando el conocimiento es el resultado de una actividad conocida como red de conocimiento.  La cantidad de mujeres en ingeniería es menor a la totalidad de los hombres en esta área en el mundo, en una investigación realizada por ACIS (Asociación Colombiana de Ingenieros de Sistemas) se descubrió que en el 2021 solo el 40\% de las mujeres se desempeña en el sector ingenieril. 

Basadas en esta cifra nace el proyecto de desarrollar para la red de conocimiento Network WE una aplicación web que permita conectar y promover la igualdad de género a más mujeres a nivel nacional e internacional en las áreas STEM (Science, Technology, Engineering and Mathematics) teniendo como foco principal de estrategia las universidad UABJB (Universidad Autónoma del Beni José Ballivian) y UDEC (Universidad de Cundinamarca).


\section{Planteamiento del problema}

Históricamente, las matemáticas, ciencia, ingeniería y la tecnología son disciplinas de estudio que se le atribuyen a grandes personajes, “hombres de ciencia”, Por ejemplo la historia de Albert Einstein y su esposa Mileva Maric relatada en el libro “The Other Einstein” en donde la autora Marie Benedict narra todos los años de…relación de la pareja, que van de 1896, cuando ambos estudiaban física en Zurich, hasta su divorcio en 1914. A lo largo de la novela se destaca la lucha constante de Mileva Maric por tener éxito en un campo de estudio dominado por los hombres. \cite{Benedict2018} Este es un factor clave de la problemática hacia el menosprecio que existió y que existe en artículos científicos de la mujer en las áreas STEM. Otros personajes como Leonardo da Vinci y Darwin dieron descubrimientos e invenciones que han aportado a la humanidad en desarrollos que cambiaron el mundo, sin embargo, el papel de la mujer en estas disciplinas no ha sido visualizado, ni reconocido en la ciencia.

Dicho esto aunque las mujeres han estado presentes en cada ambito, no se le ha visto  socialmente el reconocimiento de su conocimiento al estar este en manos masculinas evidenciando una exclusión hacia la mujer.
\section{Propuesta de solución}

\begin{itemize}
    \item Fortalece la unión y participación de mujeres en el campo multidisciplinario en las carreras enfocadas hacia la ingeniería.
    \item Apoyo en cursos tecnológicos y grupos de proyectos investigativos.
    \item Apoya el quinto objetivo de desarrollo sostenible; Igualdad de género
\end{itemize}


\section{Hipótesis}

La brecha de genero que existe al estudiar carreras enfocadas al desarrollo ingenieril requiere generar espacios de conocimiento, en este espacio, se debe de implementar y permitir la interactuaccion directamenta del conocimiento y de la experiencia para fomentar y lograr el incrementacion y participacion de mujeres en ingenieria

\section{Justificación}

Según la Organización de la Naciones Unidas) existe un enfoque basado en el planteamiento de los  Objetivos de Desarrollo Sostenible definiendo un punto fundamental en la igualdad de genero al dar a conocer que la oportunidad para enfrentar las múltiples desigualdades, es la inserción plena de las mujeres en todas las esferas es una condición indispensable para alcanzar las metas. \cite{OrganizaciondelasNacionesUnidas2016}

Razón por la cual una red de conocimiento permitirá implementar el quinto objetivo de desarrollo sostenible igualdad de género, la viabilidad de implementar este objetivo en un espacio de investigación y desarrollo multidisciplinario, hace que mediante la transferencia de conocimiento en diferentes areas de la ingeniería se creen espacios de interacción en temáticas como: brecha de género a nivel nacional, internacional y global, análisis multidisciplinario en áreas ingenieril aplicadas hacia estudiantes y profesionales, investigación en la metodología cuantitativa, empoderamiento, liderazgo, comunicación con herramientas de Las Tecnologías de la Información y las Comunicaciones, entre otras. Por ello estás áreas temáticas aplicadas en una red de conocimiento evidencian la necesidad de promover la creación de la red de mujeres, desde las universidades, contando con la participación estratégica de estudiantes universitarias de la UDEC, la UABJB, los colegios y las mujeres que se desempeñen en el ámbito laboral en áreas tecnológicas.


\section{Objetivos}

\subsection{Objetivo general}
Desarrollar de un aplicativo web para la red de conocimientos Network WE (Women Engineering) que permita fortalecer el crecimiento en la participación digital de las jóvenes en la UABJB y la UDEC

\subsection{Objetivos específicos}
  
\begin{itemize}
    \item Caracterizar la situación de género en la seccional Ubaté y la universidad autónoma del Beni José Ballivian.
    \item Definir el requerimiento funcional y no funcional de la aplicación web.
    \item Identificar los roles y comportamientos de las estudiantes de UABJB (Universidad Autónoma del Beni José Ballivian) y UDEC (Universidad de Cundinamarca).
    \item Diseñar la aplicación web de tal manera que sea intuitiva y amigable para las estudiantes que se registren e inicien sesión en la red.
\end{itemize}

\chapter{Fundamento teórico}

Para esta etapa de fundamento teorico se realizo una investigacion clave teniendo en cuenta la hipótesis y los objetivos planteados previamente para esta investigación.


\section{Marco teórico}

\section{Metodología XP}
La metodología XP es un conjunto de técnicas que ayuda a hacer los proyectos más rápido y flexible. Lo que se hace es crear un producto que le guste al cliente, por lo que es importante tener en cuenta lo que el cliente quiere. Se necesita saber exactamente lo que el cliente necesita para poder hacer lo que quiere.

\section{Análisis}
En la etapa de análisis realizamos las investigaciones y estudio de mercado de nuestra población objetivo, en este caso las mujeres estudiantes de la universidad Autónoma del Beni Jose Ballivian como las de la Universidad de Cundinamarca.
Realizamos diferentes encuestas para la tabulación de la información precisa de la cantidad de mujeres en la carrera de ingeniería de sistemas en ambas universidades, la participación de las mismas en el área STEM (Science, Technology, Engineering and Mathematics) datos específicos de la participación de estas jóvenes en desarrollo de software el lenguaje de desarrollo de su preferencia es así que obtuvimos nuestro árbol de problemas mediante el estudio de causas y efectos de la participación de las jóvenes en la industria tecnológica.

\section{Diseño}
La etapa de diseño se la desarrolla mediante los resultados de nuestro levantamiento de requerimientos y el análisis de la información recopilado en la fase anterior, una vez obtenidos estos resultados se prosiguió a el diseño de la planificación de estrategias para alcanzar los objetivos del proyecto previamente establecidos y actividades a realizarse en el transcurso de este.
\subsection{Design Thinking}
El design thinking es un método que se centra en entender y dar solución a las necesidades reales de un usuario

El proceso óptimo del método del design thinking consiste en analizar las siguientes fases en desarrollo del proyecto a entregar al usuario:

\begin{itemize}
    \item Empatía:
    
    En esta etapa 1, consiste en conocer y entender bien lo que el consumidor necesita. Para lograrlo, hay que analizar muy bien al cliente, entender lo que realmente lo motiva y sentir empatía hacia él. Para hacer esto, no es suficiente con observar al usuario, sino que también es importante hablar y trabajar con él.
    \item Definición:

    En esta etapa 2, las diferentes necesidades en la etapa anterior, se identifican cuáles son las más importantes. Esto se logra analizando cuidadosamente la gran cantidad de problemas que se detectaron anteriormente. Al determinar las necesidades más esenciales, se pueden proponer soluciones definitivas para resolverlas.
    \item Ideación:

    En esta etapa 3, el equipo de ideación debe ser creativo y proponer varias ideas para resolver los problemas específicos identificados en la fase anterior. Es importante pensar de manera libre y creativa, incluso si se cometen errores en el proceso. Para hacerlo bien, se pueden utilizar técnicas para estimular la creatividad y el pensamiento libre. 
    \item Prototipado:

    En esta etapa 4, se trata de hacer realidad las ideas seleccionadas. A veces, se puede hacer un prototipo digital, como una versión beta de un sitio web, o físico, como un dibujo o diseño. Por lo general, estos prototipos se hacen con materiales baratos, como papel, cartón o plastilina, según el presupuesto disponible.
    \item Testeo:

    En esta etapa 5, los clientes prueban y evalúan los prototipos creados anteriormente y proporcionan comentarios. El equipo puede hacer correcciones en función de estas críticas. Esta fase de validación es importante para descubrir errores y aciertos. A su vez esta etapa se enfoca en crear soluciones que se basen en cómo los usuarios reales piensan, sienten y se comportan. Permite obtener nuevos conocimientos, desarrollar nuevas formas de ver el producto y sus posibles usos, y comprender más profundamente a los usuarios y los problemas que enfrentan.

\end{itemize}
 

\section{Estado del arte}
Es importante identificar a nivel ingenieril cada uno de los antecedentes de mujeres en ingeniería en un ámbito nacional e internacional, sobre las diferentes visiones enfocadas a promover de manera estratégica a estudiantes profesionales y mujeres que se involucran de manera internacional a compartir su conocimiento. A continuación, se enuncia algunas de estas visiones encontradas en el estado de arte:


El artículo de la Cámara Colombiana de Comercio Electrónico muestra datos del Sistema Nacional de Información de la Educación Superior sobre el número de graduados en STEM por sexo en Colombia desde 2001 hasta 2020. Se destaca una brecha de género, ya que ha habido menos mujeres graduadas que hombres en estas áreas a lo largo de los años. En 2001, se graduaron 19.600 hombres y 12.092 mujeres, mientras que en 2020, 68.827 hombres se graduaron en áreas STEM en comparación con solo 38.464 mujeres.

Esto también se ve reflejado en la industria del software en Colombia, donde la falta de igualdad de oportunidades y la discriminación de género dificultan la participación de las mujeres en esta área. Las barreras incluyen estereotipos de género, falta de referentes femeninos en el sector, brechas salariales y de liderazgo, así como la falta de políticas y programas de inclusión. A pesar de estos desafíos, hay un creciente movimiento liderado por mujeres que busca derribar estas barreras y promover la igualdad de género en la industria tecnológica en Colombia.


A detalle de la investigación se descubre que Los estereotipos relacionados con los roles de género tradicionales arraigados en la sociedad colombiana, hacen parte de los principales factores que explican el bajo interés de las mujeres en las áreas STEM (Science, Technology, Engineering and Mathematics)  lo cual esta publicación demuestra una estigma social que va ascendiendo a medida que se le estipula a las niñas su rol en sociedad, aquel que va enfocado hacia las tareas del hogar y no hacia el mismo rol de capacidad intelectual de los niños, desafiando su intelecto en la ejecución de proyectos en áreas de ingenieril. 

La autora Luz Gabriela Arango Gavira  describe cómo la presencia masculina predominante en los programas universitarios de ingeniería de sistemas y en las empresas de tecnología en Colombia puede ser una barrera para la participación de mujeres en estas áreas. Esto puede generar un ambiente intimidante o poco acogedor para ellas, lo que puede limitar sus oportunidades de crecimiento profesional y su capacidad para desarrollar su potencial en el campo de la tecnología. 


\chapter{Desarrollo}


Network WE (Women Engineering) es un enfoque de desarrollo que plantea ser el espacio de interracción del conocimiento entre estudiantes de la Universidad de Cundinamarca y la Universidad Autonoma del Beni "José Ballivian", al existir un punto de conocimiento es fundamental comenzar a diseñar prototipos capaces de brindar un entorno educativo capaz de ser una de las principales fuentes de conocimiento para la red \cite{Latorre-Cosculluela2020}

\section{Metodología de la investigación / Descripción de actividades}


\section{Metodologia XP}

La metodología XP es un conjunto de técnicas que ofrecen agilidad y flexibilidad en la gestión de proyectos, esta se centra en crear un producto según las necesidades del cliente, es por ello por lo que las especificaciones del cliente están muy presentes a la hora del desarrollo del mismo es por lo cual que se debe generar un levantamiento de requerimientos específicos para el cliente.


Para la realizacion oportuna del diseño del aplicativo se uso la metodologia de desarrollo XP (Extreme Programming) en su fase II la cual consistio en:

\begin{itemize}
    \item Fase II : Simplicidad

    Al realizar el diseño del aplicarivo se recalca el diseño optimo y completo del sistema, basado desde la perspectiva que pueda lograr la comunicación con las usuarias que ingresen a la red de conocimiento Network WE (Women Engineering), para ello se toma en cuenta la filosofia de la simplicidad que se emplea en la metodologia XP: "Simplicidad en el diseño"

    La simplicidad efectua la rapidez visual del sistema siempre que sea implementado en menos tiempo  

\end{itemize}

\section{Investigacion mixta cuantitativa / cualitativa}
\section{Tabulacion de resultados encuesta 1-levantamiento de requerimientos de la aplicacion web}

En esta tabulación de información se recolectaron datos enfocados hacia el levantamiento necesario de requerimientos que se llevó a cabo en el desarrollo de aplicación web Network WE (Women Engineering) como método de comunicación internacional en las universidad UABJB (Universidad Autónoma del Beni “José Ballivian”) y UDEC (Universidad de Cundinamarca), para ello se solicitó contestar a los usuarios encuestados las siguientes preguntas más relevantes

\begin{figure}[h]
\centering
'\includegraphics{Picture1.png}'
\caption{¿Qué semestre cursas actualmente?}
\label{fig:Picture1.png}
\end{figure}

Más del 35 porciento de las estudiantes mujeres encuestadas pertenecen a VII semestre académico 

\subsection{Encuesta 1 - Encuesta 2}

Encuesta de percepción hacia el gusto visual de acceso hacia enfocado hacia el usuario cuando entra por primera vez a una aplicación web


\begin{figure}[h]
\centering
'\includegraphics{Picture2.png}'
\caption{¿Qué es lo que más te llama la atención al entrar a una web app?}
\label{fig:Picture2.png}
\end{figure}


De los 11 usuarios encuestados para el desarrollo de la aplicación web como método de comunicación se logró identificar que para el primer levantamiento de requerimiento es importante que exista una captación visual en las actividades de la red de conocimiento siendo este demandado por el 45,5%


\subsection{Encuesta 1 - Pregunta 3}

En esta pregunta se valoriza la percepción el dato más que los usuarios consideran al momento de compartir su información en la sección contáctanos cumpliendo con uno de los objetivos específicos Desarrollar un canal de comunicación que permitan dar a conocer la red de conocimiento como parte de una estrategia para promover internacionalmente la igualdad de género en UDEC (Universidad de Cundinamarca) y UABJB (Universidad Autónoma del Beni José Ballivian)

\begin{figure}[h]
\centering
'\includegraphics{Picture3.png}'
\caption{¿Qué factor consideras el más importante en la sección “Contáctenos”?}
\label{fig:Picture3.png}
\end{figure}

Se logró identificar que el factor más importante en la sección de contáctenos es el correo dado que el 63.6 porciento de los usuarios encuestados seleccionó esta opción.


\subsection{Encuesta 1 - Pregunta 8}

En el desarrollo de la web 3.0 la evolución se ha notado por ser más visual que en las anteriores versiones de 1.0 y 2.0, es por ello que solicitamos cuál de las opciones entre texto, imágenes y videos les agradaba más al interactuar con el aplicativo web.

\begin{figure}[h]
\centering
'\includegraphics{Picture4.png}'
\caption{¿Qué te gusta ver más en una web app texto, imágenes, vídeos?}
\label{fig:Picture4.png}
\end{figure}

Un factor importante que se identificó conocer en los resultados de esta pregunta fue el gusto hacia una aplicación web con una mayor caracterización visual en el desarrollo

\subsection{Encuesta 1 - Pregunta 9}

Para esta pregunta el objetivo fue conocer cual es la información que las estudiantes no desean compartir al momento de llenar sus respectivos datos dentro del desarrollo del software.

\begin{figure}[h]
\centering
'\includegraphics{Picture 5.png}'
\caption{¿Qué información no te gusta compartir?}
\label{fig:Picture5.png}
\end{figure}

Descubrimos en esta pregunta para los requerimientos de datos que serán almacenados mediante MySQL que el 44.6 de las encuestadas no desean compartir su información referente a Dirección y Teléfono. Gracias a este primer levantamiento de requerimientos aplicando la técnica de agilidad y flexibilidad de XP identificamos que lo más importante para nuestros usuarios es la estructura de la aplicación web

\subsection{Encuesta 2 - Pregunta 1}

En la Universidad del Beni – José Ballivian es la primera vez que se realiza una caracterización hacía estudiantes mujeres de ingeniería de sistemas enfocada hacia una red de conocimiento por ello, el primer paso fue conocer en qué semestre ingresaron dichas estudiantes descubriendo que:

\begin{figure}[h]
\centering
'\includegraphics{Uno.png}'
\caption{¿En qué gestión y periodo ingresaste a la UABJB?}
\label{fig:Uno.png}
\end{figure}

De las 15 estudiantes encuestadas más del 33.3 ingresaron en el periodo de año académico 2021

\section{Design Thinking}

La herramienta de design thinking es capaz de obtener resultados capaces de alcanzar los objetivos del proyecto, para ello se inicio con la primera estructuración del diseño:

 \begin{figure}[h]
\centering
'\includegraphics{mockup1 (1).png}'
\caption{Estructuración inicial fisica, desarrollo del aplicativo}
\label{fig:Picture1.png}
\end{figure}


\chapter{Resultados}

\chapter{Conclusiones}

\backmatter
\addcontentsline{toc}{chapter}{Referencias}
\bibliographystyle{IEEEtran}
\bibliography{IEEEabrv,referencias.bib}

\appendix
\chapter{Anexos}
\label{sec:anexos}

\end{document}
